% !Mode:: "TeX:UTF-8"
%% 请使用 XeLaTeX 编译本文.
% \documentclass{WHUBachelor}% 选项 forprint: 交付打印时添加, 避免彩色链接字迹打印偏淡. 即使用下一行:
 \documentclass[forprint]{WHUBachelor}
%---------------------这里添加所需的package--------------------------------
\usepackage{url}

%--------------------------------------------------------------------------
\makeatletter
\def\BState{\State\hskip-\ALG@thistlm}
\makeatother
\begin{document}
%-----------------------------------------------------------------------------

%%%%%%% 下面的内容, 据实填空.

\Ccoursename{操作系统专题实验报告} %课程名称
\title{yet another template} % useless
%\author{杨雪念} 
\Csupervisor{杨雪念} % 学生姓名
%\CsupervisorAnother{无} %指导教师二姓名、职称
\CstudentNum{2160901020} %学号
\Cclass{计算机82班} % 专业名称
%\Cschoolname{计算机学院} % 学院名
\date{2020年11月15日} % 日期

%-----------------------------------------------------------------------------

\pdfbookmark[0]{封面}{title}         % 封面页加到 pdf 书签
\maketitle
\frontmatter
\pagenumbering{Roman}              % 正文之前的页码用大写罗马字母编号.2019.6.16:更新 正文之前的页码隐藏,无需显示
%-----------------------------------------------------------------------------
%\include{includefile/frontmatter}    % 加入摘要, 申明.
%==========================把目录加入到书签==============================%%%%%%


\tableofcontents
\thispagestyle{empty}				%不显示罗马数字 ——zmx更新于2019.06.18
\addtocontents{toc}{\protect\thispagestyle{empty}}




\mainmatter %% 以下是正文
%%%%%%%%%%%%%%%%%%%%%%%%%%%--------main matter-------%%%%%%%%%%%%%%%%%%%%%%%%%%%%%%%%%%%%
\pagestyle{plain}%plain
%\cfoot{\thepage{\zihao{5}\bf\usefonttimes}}
%\renewcommand{\baselinestretch}{1.6}
%\setlength{\baselineskip}{23pt}
\baselineskip=23pt  % 正文行距为 23 磅

%此处书写正文-------------------------------------------------------------------------------------


\chapter{先说重要的}
 
 \section{具体使用步骤}

 \begin{description}

  \item[Step 1]  进入 includefile 文件夹,  打开 frontmatter.tex, backmatter.tex 这两个文档,
        分别填写 (1) 中文摘要, (2) 实验结论.

  \item[Step 2]  打开主文档 Experiment-template.tex, 填写题目、学生姓名等等信息, 书写正文.

  \item[Step 3]  使用 XeLaTeX 编译. 具体见 \ref{sec-compile} 节.


\end{description}



\section{编译的方法}\label{sec-compile}

默认使用 XeLaTeX 编译, 直接生成~pdf 文件.

若另存为新文档, 请确保文档保存类型为 \verb|:UTF-8|. 当然目前很多编辑器默认文字编码为 UTF-8.
WinEdt 9.0 之后的版本都是默认保存为 UTF-8 的.


%使用~XeLaTeX 编译, 直接生成~pdf 文件.
%pdf 文件也可以反向搜索! \CJKunderwave{双击~pdf 中要修改的文字, 将直接跳转到源文件中相应位置.}
%





\section{文档类型选择}

{\textbf{\zihao{-2}{本小节是毕业论文打印介绍,实验报告可以略过}}}

文档类型有 2 种情形:

\begin{table}[ht]\centering
\begin{tabular}{ll}
\hline
   \verb|\documentclass{WHUBachelor}|                     &  毕业论文 \\
   \verb|\documentclass[forprint]{WHUBachelor}|        &  毕业论文打印版 \\
\hline
\end{tabular}
\end{table}
相关解释见下节.


\section{打印的问题}

{\textbf{\zihao{-2}{本小节是毕业论文打印介绍,实验报告可以略过}}}

\begin{enumerate}[i)] 
%  \item  论文要求\colorbox{yellow}{单面打印}.
  \item  关于文档选项 forprint: 交付打印时, 建议加上选项 forprint, 以消除链接文字之彩色, 避免打印字迹偏淡.
  \item  打印时留意不要缩小页面或居中. 即页面放缩方式应该是``无''(Adobe Reader XI 是选择``实际大小'').
           有可能页面放缩方式默认为``适合可打印区域'', 会导致打印为原页面大小的 $97\%$.
           文字不要居中打印, 是因为考虑到装订, 左侧的空白留得稍多一点(模板已作预留).
  \item  遗留问题: 封面需要打印部重新制作.  校内打印部通常有现成的模板.
           我们自己做的封面, 打印部不一定好用.
\end{enumerate}
%如果不是彩色打印机, 请在打印时, 选择将彩色打印为黑白, 否则彩色文字打出的墨迹会偏淡.

\textbf{问}: {\kaishu 生成 PDF 文件时,不能去掉目录和文章的引用彩色方框,请问怎么解决?}

\textbf{答}: {\kaishu 方框表示超级链接, 只在电脑上看得见. 实际打印时, 是没有的. 另外, 文档类型加选项 forprint 之后, 这些框框会隐掉的. }

 \vfill

本文档下载更新地址: \url{https://github.com/xiaoxinganling/WHUExperiment}. 使用之前, 请移步查看是否有更新.

问题反馈及建议, 请联系: mxzhou1998@gmail.com.



\chapter{杂七杂八的话}

\section{Readme}

模板文件的结构, 如下表所示:
 \begin{table}[ht]\centering
\begin{tabular}{r|r|l}
	\hline\hline
	\multicolumn{2}{l|}{Experiment-template.tex }       & 主文档. 在其中填写正文.             \\ \hline
	                                & frontmatter.tex & 郑重声明、摘要.               \\ \cline{2-3}
	\raisebox{1em}{includefile 文件夹} &  backmatter.tex & 实验结论.                       \\ \hline
	\multicolumn{2}{l|}{figures 文件夹}                  & 存放图片文件.                   \\ \hline
	\multicolumn{2}{l|}{WHUBachelor.cls }             & 定义文档格式的 class file. 不可删除. \\ \hline\hline
\end{tabular}
\end{table}

无需也不要改变、移动上述文档的位置.

如果不习惯用~\verb|\include{ }|~的方式加入``子文档'', 当然可以把它们合并在主文档, 成为一个文档.
({\kaishu 但是这样并不会给我们带来方便.})




 \section{字体调节}

\begin{tabular}{ll}
	\verb|\songti|   & {\songti 宋体}   \\
	\verb|\heiti|    & {\heiti 黑体}    \\
	\verb|\fangsong| & {\fangsong 仿宋} \\
	\verb|\kaishu|   & {\kaishu 楷书}
\end{tabular}


\section{字号调节}
字号命令: \verb|\zihao| \index{zihao}

\begin{tabular}{ll}
\verb|\zihao{0}| &\zihao{0}  初号字 English \\
\verb|\zihao{-0}|&\zihao{-0} 小初号 English \\
\verb|\zihao{1} |&\zihao{1}  一号字 English \\
\verb|\zihao{-1}|&\zihao{-1} 小一号 English \\
\verb|\zihao{2} |&\zihao{2}  二号字 English \\
\verb|\zihao{-2}|&\zihao{-2} 小二号 English \\
\verb|\zihao{3} |&\zihao{3}  三号字 English \\
\verb|\zihao{-3}|&\zihao{-3} 小三号 English \\
\verb|\zihao{4} |&\zihao{4}  四号字 English \\
\verb|\zihao{-4}|&\zihao{-4} 小四号 English \\
\verb|\zihao{5} |&\zihao{5}  五号字 English \\
\verb|\zihao{-5}|&\zihao{-5} 小五号 English \\
\verb|\zihao{6} |&\zihao{6}  六号字 English \\
\verb|\zihao{-6}|&\zihao{-6} 小六号 English \\
\verb|\zihao{7} |&\zihao{7}  七号字 English \\
\verb|\zihao{8} |&\zihao{8}  八号字 English \\
\end{tabular}

\section{已加入的常用宏包}

\begin{description}
%  \item[amsmath,amssymb]
  \item[cite]  参考文献引用, 得到形如 [3-7] 的样式.
  \item[color,xcolor]  支持彩色.
  \item[enumerate]  方便自由选择 enumerate 环境的编号方式. 比如

  \verb|\begin{enumerate}[(a)]| 得到形如 (a), (b), (c) 的编号.


  \verb|\begin{enumerate}[i)]| 得到形如 i), ii), iii) 的编号.

  \verb|\begin{enumerate}[\hspace{1cm}(1)]| \verb|\hspace|命令用于调整距离

\end{description}

另外要说明的是,  itemize, enumerate, description 这三种 list 环境, 已经调节了其间距和缩进,
以符合中文书写的习惯.

\section{标点符号的问题}

建议使用半角的标点符号, 后边再键入一个空格. 特别是在英文书写中要注意此问题!

双引号是由两个左单引号、两个右单引号构成的: \verb|``  ''|. 左单引号在键盘上数字~1 的左边.

但是, 无论您偏向于全角或半角, 强烈建议您使用实心的句号, 只要您书写的是自然科学的文章.
原因可能是因为, 比如使用全角句号的句子结尾处的``$x$。''容易误为数学式~$x_0$(\verb|$x_0$|)吧.



\section{引用的问题}


\subsection{参考文献的引用}

参考文献的引用, 用命令~\verb|\cite{ }|. 大括号内要填入的字串, 是自命名的文献条目名.

比如, 通常我们会说:

 {\kaishu
关于此问题, 请参见文献 \cite{r2}. 作者某某还提到了某某概念\upcite{r1}.}


上文使用的源文件为:

 {\kaishu
关于此问题, 请参见文献~\verb|\cite{r2}|. 作者某某还提到了某某概念~\verb|\upcite{r1}|.
}

其中~\verb|\upcite| 是自定义命令, 使文献引用呈现为\CJKunderdot{上标形式}.

({\heiti 注意:} {\kaishu 这里文献的引用, 有时需要以上标形式出现, 有时需要作为正文文字出现, 为什么?})

另外, 要得到形如~\cite{r1,r3,r4,r5} 的参考文献连续引用, 需要用到 cite 宏包(模板已经加入),
在正文中使用~\verb|\cite{r1,r3,r4,r5}| 的引用形式即可.
或者, 连续引用的上标形式: 使用~\verb|\upcite{r1,r2,r3}|, 得到\upcite{r1,r2,r3}.

\subsection{定理和公式的引用}

\begin{theorem}[谁发现的]\label{th-abcd}
最大的正整数是~$1$.
\end{theorem}

\begin{proof}
要找到这个最大的正整数, 我们设最大的正整数为~$x$, 则~$x \geqslant 1$, 两边同时乘以~$x$, 得到
\begin{equation}\label{eq-abc}
x^2 \geqslant x.
\end{equation}
而~$x$ 是最大的正整数, 由~\eqref{eq-abc} 式得到
\[
x^2 = x.
\]
所以
\begin{equation*}
x = 1.
\end{equation*}
\end{proof}

定理~\ref{th-abcd} 是一个重大的发现.

%%%%----- 定义等环境的举例 --------
\begin{definition}[整数]
 正整数(例如 1, 2, 3)、负整数(例如 ${−1}$, $−2$, $−3$)与零(0)合起来统称为{\heiti 整数}.
\end{definition}

\begin{remark}
  整数集合在数学上通常表示为 $\mathbf{Z}$ 或 $\mathbb{Z}$, 该记号源于德语单词 Zahlen(意为``数'')的首字母.
\end{remark}

\begin{proposition}
任意两个整数相加、相减、相乘的结果, 仍然是整数.
\end{proposition}

\begin{example}
  $1+2=3$.
\end{example}

\begin{corollary}
   在整数集合内, 相加、相减、相乘运算是封闭的.
\end{corollary}

\section{图形与表格}

支持对~eps, pdf, jpg 等等常见图形格式.

再次\colorbox{red!45}{澄清一个误会}: \LaTeX{} 支持的图形格式绝非 eps 这一种. 无需特意把图片转化为 eps.

用形如~\verb|\includegraphics[width=12cm]{Daisy.jpg}| 的命令可以纳入图片.

如图~\ref{fig:1} 是一个纳入~jpg 图片的例子.

\begin{figure}[ht]
\centering
  \includegraphics[width=\textwidth]{Daisy.jpg}
  \caption{一个彩色 jpg 图片的例子}
  \label{fig:1}
\end{figure}

表格问题, 建议使用``三线表'', 如表 \ref{tab:1}.

\begin{table}[ht]
\centering
\caption{一般三线表}
\label{tab:1}
    \begin{tabular}{c c c c c c c c c c c}
    \hline
    123 & 4  & 5  & 123 & 4 & 5123 & 4 & 5 & 123 & 4 & 5\\
    \hline
    67 & 890 & 13 & 123 & 4 & 5123 & 4 & 5 & 123 & 4 & 5\\
    67 & 890 & 13 & 123 & 4 & 5123 & 4 & 5 & 123 & 4 & 5\\
    67 & 890 & 13 & 123 & 4 & 5123 & 4 & 5 & 123 & 4 & 5\\
    \hline
    \end{tabular}
\end{table}


%%%%============================================================================================================%%%

\chapter{其他事项}
以下是广告时间, 插播一段广告:
\begin{itemize}
    \item 插图\index{插图}的制作, 建议用 pgf, 也叫 tikz.
          pgf 的长处是源文件直接植入~\TeX~文档, 管理起来非常方便.
    这里有我写的一个关于初次使用~pgf~的帖子:\\    \url{http://bbs.ctex.org/forum.php?mod=viewthread&tid=30480}.
    \item 生成参考文献, 建议使用~BibTeX.\index{BibTeX}: \\
    \url{http://bbs.ctex.org/forum.php?mod=viewthread&tid=26056}.

          {\kaishu 使用 BibTeX{} 做参考文献时,
      借助 EndNote 或者 NoteExpress, 可以非常漂亮简单地解决 bib 文件的录入问题.
      NoteExpress 在校图书馆网站有正版软件提供下载.
      当然 EndNote 本身就是 Thomson Corporation 推出的(和 SCI 搜索引擎是同一家公司),
      和多个重要文献搜索引擎有良好的功能配合.

      Google 学术搜索也提供了文献的 bib 格式.
      录入参考文献时, 偶尔用一用 Google 学术搜索, 还可以核查或减少录入的错误, 并减少录入的工作量.}

    \item 幻灯片\index{幻灯片}的制作, 建议使用~Beamer. 这里有我写的一个模板, 谨供参考:\\
    \url{http://bbs.ctex.org/forum.php?mod=viewthread&tid=27695}.
\end{itemize}


%此处结束正文-------------------------------------------------------------------------------------------------


\include{includefile/backmatter} %%%结论

%%%============================================================================================================%%%

%%%=== 参考文献 ========%%%
\cleardoublepage\phantomsection
\addcontentsline{toc}{chapter}{参考文献}
\renewcommand{\baselinestretch}{1.6}
\begin{thebibliography}{00}

  \bibitem{mapreduce} Dean J, Ghemawat S. MapReduce: Simplified Data Processing on Large Clusters[A].Eric A. Brewer, Peter Chen.6th Symposium on Operating Systems Design and Implementation(OSDI 2004)[C], San Francisco, California, USA: {USENIX} Association, 2004:137--150.

  \bibitem{r1} 作者. 文章题目 [J].  期刊名, 出版年份,卷号(期数): 起止页码.

  \bibitem{r2} 作者. 书名 [M]. 版次. 出版地:出版单位,出版年份:起止页码.

  \bibitem{r3} 邓建松等, 《\LaTeXe~科技排版指南》, 科学出版社.

  \bibitem{r4} 吴凌云, 《CTeX~FAQ (常见问题集)》, \textit{Version~0.4}, June 21, 2004.

  \bibitem{r5} Herbert Vo\ss, Mathmode, \url{http://www.tex.ac.uk/ctan/info/math/voss/mathmode/Mathmode.pdf}.


\end{thebibliography}



%%%-------------- 附录. 不需要可以删除.-----------


\appendix

\chapter{测试}

\section{第一个测试}
测试公式编号
\begin{equation}
1+1=2.
\end{equation}

表格编号测试

\begin{table}[h]
  \centering
  \caption{测试表格}
  \begin{tabular}{*{20}c}
     \hline
     % after \\: \hline or \cline{col1-col2} \cline{col3-col4} ...
     11 & 13  & 13  & 13  & 13 \\
     12 & 14  & 13  & 13  & 13 \\
     \hline
   \end{tabular}
\end{table}


\chapter{附录测试}

%%%-------------- 教师评语评分 ------------------
\begin{teacher}
\thispagestyle{empty}
评语: 
\par
\vspace*{12.5cm}
\hspace*{7.5cm}评分: 
\vspace*{1cm}

\hspace*{7.3cm}评阅人:

\vspace*{0.5cm}

\hspace*{10.1cm}年\hspace*{1cm}月\hspace*{1cm}日

\vspace*{0.5cm}

{\songti \zihao{4} \makebox[1cm][s]{(备注:对该实验报告给予优点和不足的评价,并给出百分制评分。)}}

\end{teacher}


\cleardoublepage
\end{document}





