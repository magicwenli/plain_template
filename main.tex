% !Mode:: "TeX:UTF-8"
%% 请使用 XeLaTeX 编译本文.
% \documentclass{WHUBachelor}% 选项 forprint: 交付打印时添加, 避免彩色链接字迹打印偏淡. 即使用下一行:
 \documentclass[forprint]{WHUBachelor}
%---------------------这里添加所需的package--------------------------------
\usepackage{url}

%--------------------------------------------------------------------------
\makeatletter
\def\BState{\State\hskip-\ALG@thistlm}
\makeatother
\begin{document}
%-----------------------------------------------------------------------------

%%%%%%% 下面的内容, 据实填空.

\Ccoursename{操作系统专题实验报告} %课程名称
\title{yet another template} % useless
%\author{杨雪念} 
\Csupervisor{杨雪念} % 学生姓名
%\CsupervisorAnother{无} %指导教师二姓名、职称
\CstudentNum{2160901020} %学号
\Cclass{计算机82班} % 专业名称
%\Cschoolname{计算机学院} % 学院名
\date{2020年11月15日} % 日期

%-----------------------------------------------------------------------------

\pdfbookmark[0]{封面}{title}         % 封面页加到 pdf 书签
\maketitle
\frontmatter
\pagenumbering{Roman}              % 正文之前的页码用大写罗马字母编号.2019.6.16:更新 正文之前的页码隐藏,无需显示
%-----------------------------------------------------------------------------
%\include{includefile/frontmatter}    % 加入摘要, 申明.
%==========================把目录加入到书签==============================%%%%%%


\tableofcontents
\thispagestyle{empty}				%不显示罗马数字 ——zmx更新于2019.06.18
\addtocontents{toc}{\protect\thispagestyle{empty}}




\mainmatter %% 以下是正文
%%%%%%%%%%%%%%%%%%%%%%%%%%%--------main matter-------%%%%%%%%%%%%%%%%%%%%%%%%%%%%%%%%%%%%
\pagestyle{plain}%plain
%\cfoot{\thepage{\zihao{5}\bf\usefonttimes}}
%\renewcommand{\baselinestretch}{1.6}
%\setlength{\baselineskip}{23pt}
\baselineskip=23pt  % 正文行距为 23 磅

%此处书写正文-------------------------------------------------------------------------------------


\chapter{openEuler系统环境实验}
 
 \section{实验目的}

 \section{实验内容}

 \section{实验思想}

 \section{实验步骤}

 \section{测试数据设计}
 
 \section{程序运行初值及运行结果分析}
 
 \section{实验总结}
 
 \subsection{实验中的问题与解决过程}
 
 \subsubsection{问题描述}
 
 \subsubsection{解决过程}
 
 \subsection{实验收获}
 
 \subsection{意见与建议}
 

%此处结束正文-------------------------------------------------------------------------------------------------


%\include{includefile/backmatter} %%%结论

%%%============================================================================================================%%%

%%%=== 参考文献 ========%%%
%\cleardoublepage\phantomsection
%\addcontentsline{toc}{chapter}{参考文献}
%\renewcommand{\baselinestretch}{1.6}
%\begin{thebibliography}{00}

%  \bibitem{mapreduce} Dean J, Ghemawat S. MapReduce: Simplified Data Processing on Large Clusters[A].Eric A. Brewer, Peter Chen.6th Symposium on Operating Systems Design and Implementation(OSDI 2004)[C], San Francisco, California, USA: {USENIX} Association, 2004:137--150.
%
%  \bibitem{r1} 作者. 文章题目 [J].  期刊名, 出版年份,卷号(期数): 起止页码.
%
% \end{thebibliography}



%%%-------------- 附录. 不需要可以删除.-----------


\appendix

\chapter{附件}

\section{第一个测试}
测试公式编号
\begin{equation}
1+1=2.
\end{equation}

表格编号测试

\begin{table}[h]
  \centering
  \caption{测试表格}
  \begin{tabular}{*{20}c}
     \hline
     % after \\: \hline or \cline{col1-col2} \cline{col3-col4} ...
     11 & 13  & 13  & 13  & 13 \\
     12 & 14  & 13  & 13  & 13 \\
     \hline
   \end{tabular}
\end{table}


\chapter{附录测试}




\cleardoublepage
\end{document}





